\documentclass[a4paper, landscape]{scrreprt}
\KOMAoptions{DIV=30} % Parameter mit dem man den Seitenrand ändern kann.

\def\jafp#1{
    \begin{center}
    \includegraphics[width=1.0\textwidth]{#1}
    \end{center}
}
\usepackage{graphicx}
\graphicspath{{./images/}}


\begin{document}
\section*{a) Zustaende}
\jafp{1s0}
\jafp{2p-1}
\jafp{2p0}
\jafp{2p1}

\section*{b) Oszillationen}
\jafp{osc_-1_0}
\jafp{osc_-1_1}
\jafp{osc_-1_2}
\jafp{osc_-1_3}
\jafp{osc_-1_4}
\jafp{osc_-1_5}
\jafp{osc_-1_6}

\jafp{osc_0_0}
\jafp{osc_0_1}
\jafp{osc_0_2}
\jafp{osc_0_3}
\jafp{osc_0_4}
\jafp{osc_0_5}
\jafp{osc_0_6}

\jafp{osc_1_0}
\jafp{osc_1_1}
\jafp{osc_1_2}
\jafp{osc_1_3}
\jafp{osc_1_4}
\jafp{osc_1_5}
\jafp{osc_1_6}

\section*{c) koppelnde Polarisationen}

Bei dem \"Ubergang $1S_0$ nach $2P_0$ erkennt man einen Dipol in Z-Richtung, welcher im Verlauf der Oszillation seine Polarit\"at wechselt. Es liegt nahe, dass dieser \"Ubergang mit linear Polarisiertem Licht koppelt.

Bei den beiden anderen \"Uberg\"angen bildet sich ein Dipol aus, welcher in der X-Y Ebene liegt und im Verlauf der Oszillation rotiert. Dies entspricht einer Kopplung mit zirkul\"ar polarisiertem Licht.

Der $1S_0$ nach $2P_1$ \"Ubergang dreht sich im zeitlichen Verlauf rechtsherum rotiert, der $1S_0$ nach $2P_{-1}$ \"Ubergang rotiert in links Richtung.

Dies legt nahe, dass bei $1S_0$ nach $2P_1$ rechtszirkul\"ar polarisiertes Licht den \"Ubergang herbeif\"uhrt. Analog sollte der Zerfall $2P_1 \rightarrow 1S_0$ linkszirkul\"ar polarisiertes Licht emittieren (quasi ja Absorbtion die r\"uckw\"arts abl\"auft xD).

Der $1S_0 \rightarrow 2P_{-1}$ \"Ubergang wird also durch linkszirkul\"ar polarisiertes Licht hervorgerufen, der Zerfall emittiert rechtszirkul\"ares.

\end{document}
